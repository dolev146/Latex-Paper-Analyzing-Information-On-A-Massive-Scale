\documentclass[12pt]{article}
\usepackage{amsmath,amsfonts,amssymb}
\usepackage{geometry}
\usepackage{pifont}
\usepackage[utf8]{inputenc}
\usepackage[english]{babel}
\usepackage{ragged2e}
\usepackage{blindtext}

\geometry{a4paper, margin=1in}

\title{\textbf{Analysis of High-Dimensional Data}}
\author{
    Dolev Dublon\\
    Moria Grohar\\
    Shirel Zecharia\\
    Roy Harel\\
    Hadas Evers
}
\date{\today}

\begin{document}

\maketitle

\begin{abstract}
    This Survey explores the fascinating world of high-dimensional data analysis, focusing on the properties of random matrices. These matrices play a crucial role in various fields such as computer science, statistics, and machine learning, especially concerning their singularity and the analysis of their smallest singular values.
\end{abstract}

\section{Introduction}

In the realm of high-dimensional data analysis, random matrices have captivated the attention of mathematicians and scientists due to their ubiquitous presence in various domains, including computer science, statistics, and machine learning. Understanding their behavior is paramount, particularly regarding their singularity (non-invertibility) and the analysis of their smallest singular values. This article delves into ten research papers that shed light on these aspects.

\section{Mathematical Context - Main}

The article "Random Symmetric Matrices Are Almost Surely Non-Singular"
by K. Costello, T. Tao, and V. Vu presents a significant result in the
field of random matrices. The main finding of the article is the proof
that a random symmetric matrix $ Q_n $ with independent and identically distributed 
(with the same distribution) Bernoulli variables as its upper diagonal entries
is almost surely non-singular, with a probability of $ 1-O(n^{-1/8+\delta}) $  for any $ \delta > 0 $.
This result extends previous results for random matrices to more general models of random matrices.
The article presents the history of the non-singularity problem in random matrices,
namely whether it is true that a random matrix $ A_n $ with independent Bernoulli variables
is almost surely non-singular. This question was positively answered by Komlós in 1967,
and later he generalized the result to more general models of random matrices. In a recent paper,
Tao and Vu found a different proof for random matrices that 
provides a precise estimate for the absolute value of the determinant of the matrix $ A_n $.
Building upon these previous proofs, the authors develop a quadratic version of 
Littlewood-Offord type results concerning the concentration of random variables to prove 
the non-singularity of $ Q_n $- a  random symmetric matrix.
This method allows researchers to overcome the challenge of the row and column 
transpose, which was a hurdle in previous proofs for random matrices due to the 
dependence between the row vectors of the matrix $ Q_n $.
The article raises open questions for future research in the field of random matrices:

Determinant Estimation: The article raises the question of estimating the 
determinant of random matrices. The estimation provided in the article is: $ |det\ {Q_n|=}n^{\left(1/2-o\left(1\right)\right)n} $.

Singularity Probability: Another open question raised in the article relates to estimating the 
probability that a random matrix is singular. The authors estimate that
the probability of $ Q_n $ being a singular matrix is $ {(1/2+o\left(1\right))}^n $.

The quadratic variant of the Littlewood-Offord :
Let Q be a quadratic random variable defined as: $ Q=\sum_{1\le i,j\le n}{c_{ij}z_iz_j} $  where $ z_i $
are random variables, $ {1,\ldots,n}=\ U_1\cup U_2  $
is a non-trivial partition, and S is a non-empty subset of
$ U_1 $. For each $ i\in S $, let  $d_i$  be the number of indices  $j\in U_2$ 
such that $|c_{ij}|\geq1$. If 
$ d_i\geq1$ for each $i\in S$, and I is an interval of length 1, then:
$ P\left(Q\in I\right)=O{({|S|}^{-1/2}+{|S|}^{-1}\sum_{i\in S}{d_i}^{-1/2})}^{1/4} $


\section{roy article1 summary}

\section{roy article2 summary}

\section{shirel Singularity of random symmetric matrices revisited summery}
\begin{flushleft}
Singularity of Random Symmetric Matrices: This paper studies the probability, denoted as ${P(det(M_n) = 0)}$, of a singular random ${n \times n}$ matrix ${M_n}$ drawn uniformly from matrices with entries of ${-1}$ and ${1}$. It's a long-standing problem with the conjecture that ${P(det(M_n) = 0)}$ goes to zero exponentially fast with ${n}$, written as ${P(det(M_n) = 0) = (1 + o(1))n^{2}2^{-n+1})}$. Prior work established bounds on this probability but couldn't overcome a natural barrier of ${exp(-c\sqrt{n} log n)}$ for some constant ${c}$, where the randomness in the matrix isn't "reused."\\
This paper breaks the barrier by introducing a "rough" inverse Littlewood-Offord theorem, proving that\ ${P(det(M_n) = 0) \leq exp(-cpn log n)}$ for some constant ${c}$ and sufficiently large ${n}$. The authors build upon previous work that divides vectors ${v}$ into structured and unstructured and analyzes their contribution to ${P(det(M_n) = 0)}$. Their key improvement lies in a simpler and stronger "rough" inverse Littlewood-Offord theorem, which defines concepts like ${N_{\mu}(w) := {x \in \mathbb{Z}_{p} : P(X \mu (w) = x) > 2^{-1}P(X_{\mu}(w) = 0)}}$ to analyze the "neighborhood" of a vector ${w}$ under a random walk ${X_{\mu}(v) := \varepsilon_1 v_1 + ... + \varepsilon_n v_n}$, where ${\varepsilon_i}$ are independent and take values ${-1}$, ${0}$, or ${1}$ with equal probability ${\mu /2}$.
\end{flushleft}

\section{moria Singularity of random symmetric matrices revisited summery}

\section{moria Singularity of discrete random matrices summery}

\section{shirel On the smoothed analysis of of the smallest singular value with discrete noise}

\section{dolev article1 summery of graph}

\section{shirel on graph article}

\section{Haddas last article}

\section{Results}
// in the mean time we don't fill this

\end{document}
