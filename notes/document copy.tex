\documentclass{article}
\usepackage{amssymb}
\usepackage[left=20mm,right=20mm,top=10mm,bottom=17mm, paper=a4paper]{geometry}

\title{SINGULARITY OF DISCRETE RANDOM MATRICES - Summary}
\author{Moria Grohar}
\date{\today}

\begin{document}

\maketitle


\section{Basic Problem}
\textbf{Singularity of random Bernoulli matrices:} \\
Let ${M_n}$ be an n $\times$ n random matrix with each entry an independent $Ber(1/2)$ random variable: \\
Estimate $ q_n := \mathbb{P}[M_n$ is singular$]=\mathbb{P}[$zero row or column$]+ (1+o_n(1))\mathbb{P}[$two equal (up to sign) rows or columns$]$
thereby confirming a folklore conjecture.
\subsection{First main result:}
\label{sec:first_main_results}

Let $\xi$ be a discrete random variable, let $s_n(M_n)$ and let $M_n(\xi)$ be an n X n random matrix whose entries are independent copies of $\xi$. Then:

$\mathbb{P}[M_n$ is singular$]= (1+o_n(1))(2n\mathbb{P}[\mathcal{E}{e1}]+n(n-1)\mathbb{P}[\mathcal{E}_{e1-e2}]+n(n-1)\mathbb{P}[\mathcal{E}_{e1+e2}])$\\
upper bound:
$$\mathbb{P}[s_n(M_n(\xi)) \leq t/\sqrt{n}] \leq C_\xi t+2n\mathbb{P}[\mathcal{E}_{e1}]+(1+\epsilon)^n\mathbb{P}[\mathcal{E}_{e1-e2}].$$

upper bound for non-uniform discrete distributions is strictly stronger:
$$\mathbb{P}[s_n(M_n(\xi)) \leq t/\sqrt{n}] \leq C_\xi t+2n\mathbb{P}[\mathcal{E}_{e1}]+(1+O(\exp(-c_\xi n)))(n(n-1)\mathbb{P}[\mathcal{E}_{e1+e2}]+n(n-1)\mathbb{P}[\mathcal{E}_{e1-e2}]).$$
Lower bound:  $ q_n \geq (n^2 + n) \cdot 0.5^n $\\
\subsection{Previous work:}

Komlós (1967): $q_n = o_n(1)$ \\
Kahn-Komlós-Szemerédi (1995): $q_n \leq (0.999+o_n(1))^n$ \\
Tao-Vu (2006, 2007): $q_n \leq (0.75 + o_n(1))^n$ \\
Kahn-Komlós-Szemerédi (2010): $q_n \leq (1/\sqrt{2} +o_n(1))^n$ \\
Tikhomirov (2018): $q_n =(0.5 +o_n(1))^n$ \\

\section{Previous Results}
\subsection{Strong invertibility of sparse Bernoulli matrices:}
Let ${M_n}$ be an n $\times$ n random matrix with each entry an independent $Ber(\rho)$ random variable for $\rho \in (0, 1/2)$: \\
Conjecture (Folklore):

$q_n = (1 + o_n(1)) \cdot \mathbb{P}[$zero row or column$] = (1 + o_n(1)) \cdot 2n(1-p)^n$ \\
\\\textbf{Works in the past about this problem:} \\
Previous studies on singularity probability achieved suboptimal results or only applied to specific distributions.\\

Basak and Redelson (2018): ${\rho}_n \in n^{-1} \cdot (\log{n} \pm \omega(1))$.

Litvak and Tikhomirov (2020): ${Cn^{-1} \cdot \log{n} \leq {\rho}_n \leq c}$.

Huang (2020): ${n^{-1} \cdot \log{n} \leq {\rho}_n \leq \omega(n^{-1} \cdot \log{n})}$.\\
What about  $\rho \in (0, 1/2)$?

Tikhomirov (2018): ${\rho}_n = ( 1 - \rho + o_n(1))^n$. - best results \\
Before this work, no analogous result existed.\\
This paper significantly improves these limitations:

$\bullet$ {Solves \ref*{sec:first_main_results} for a wider range of sparse Bernoulli distributions.}

$\bullet$ {Obtains the correct base of the exponent for general discrete distributions (unlike prior work).}

Finally, it mentions potential issues with a recent claim by another researcher on a related topic.

\subsection{Second main result:}
\label{sec:second_main_results}

Fix a discrete distribution $\xi$. There exist $\delta, \rho, \eta > 0$ depending on $\xi$ such that for all sufficiently large $n$ and $t \leq 1$,
$$\mathbb{P}[\inf_{x\in Cons(\delta,\rho)}||M_n(\xi)x||_2\leq t] \leq n\mathbb{P}[\mathcal{E}_{e1}] + \left( \begin{array}{c} n \\ 2 \end{array} \right)(\mathbb{P}[\mathcal{E}_{e1-e2}]+\mathbb{P}[\mathcal{E}_{e1+e2}]) + (t + \mathbb{P}[\mathcal{E}_{e1-e2}])e^{-\eta n}.$$
the set $Cons(\delta,\rho)$ appearing above is the set of unit vectors which have at least $(1-\delta)n$ coordinates within distance $\rho/\sqrt{n}$ of each other.
\section{General Setup}
\subsection{Singularity of discrete random matrices:}
Definition (Discrete random variable): A discrete random variable $\xi$ is a real-valued, non-constant, finitely supported random variable.

Estimate $ q_n(\xi) := \mathbb{P}[M_n(\xi)$ is singular$] $.\\
$ q_n(\xi) := (\alpha(\xi) + o_n(1))^n$ known for some special choices of $\xi $.\\
$\bullet$ {Tikhomirov (2018): $ \xi = Ber(\rho) , \rho \in (0, 1/2]$}.\\
$\bullet$ {Bourgain-Vu-Wood (2010): $ \xi = \pm 1$ w.p. 1/4, 0 w.p. 1/2 etc}.\\
$\bullet$ {Notably, $ \xi = Ber(\rho), \rho \in (1/2, 1)$ was open}.\\

\section{Conjecture}
\subsection{Strong Singularity of discrete random matrices:}
Conjecture (Folklore) - \textbf{first term}:

$q_n = (1 + o_n(1))(\mathbb{P}[$zero row / column$] + \mathbb{P}[$two equal / opposite rows / columns$])$. \\
Theorem (J.-Sah-Sawhnwy, 2020):

For $\xi$ not uniform on its support,

$q_n(\xi) = \mathbb{P}[$zero row / column$] + (1 + O(e^{-c_{\xi}n}))\mathbb{P}[$two equal / opposite rows / columns$]$. \\
$\bullet$ {For $ \xi = Ber(\rho), \rho in (0, 1/2)$, get \textbf{first two} terms of the expansion}:

$q_n(\xi) = 2n(1-\xi)^n + (n^2 - n)(\xi^2 + (1- \xi)^2)^n + ...$\\
$\bullet$ {For $ \xi = Ber(\rho), \rho in (1/2, 1)$, get the first term of the expansion}:

$q_n(\xi) = (n^2 - n)(\xi^2 + (1- \xi)^2)^n + ...$

\subsection{Singularity of combinatorial random matrices:}

Let ${M_n}$ be an n $\times$ n matrix, each row is independently sampled from $\{0, 1\}^n_{\lfloor n/2\rfloor}$.\\
Conjecture (Nguyen, 2011):

$q_n := \mathbb{P}[M_n$ is singular$] = (1 + o_n(1))^n $. \\
$\bullet$ {Nguyen (2011): $ q_n \leq O_c(n^{-c})$ for any $c > 0$}.\\
$\bullet$ {Ferber-J.-Luh-Samotij (2019): $ q_n \leq C \exp (-n^{c})$}.\\
$\bullet$ {Tran (2020): $ q_n \leq  C \exp (-cn)$}.\\
Theorem (J.-Sah-Sawhnwy, 2020): $q_n = (1.2 + O_n(1))^n$. \\

\section{Proof of Singularity}
\subsection{Littlewood-Offord theory:}
The Littlewood-Offord Problem (1943):

Let $a_1, ..., a_n$ be a nonzero integers, and let $\epsilon_1,...,\epsilon_n$ be i.i.d. Rademacher random cariables.

For a given $x \in \mathbb{Z}$, how big can $\mathbb{P}[\epsilon_1a_1+\cdot \cdot \cdot +\epsilon_na_n=x]$ be?\\
Let $a=(a_1, ..., a_n)$ and define its atom probability to be:

$\rho(a):= \sup_{x \in \mathbb{Z}}\mathbb{P}[\epsilon_1a_1+\cdot \cdot \cdot +\epsilon_na_n=x]$.\\
With the help of this equation, We can find the minimum value for some input.\\
Examples:

$\bullet$ {If $ a=(10,100,1000,...,10^n)$, then $\rho(a) = 2^{-n}$}.

$\bullet$ {If $ a=(1,...,1)$, then $\rho(a) = 2^{-n}\left( \begin{array}{c} n \\ {\lfloor n/2 \rfloor} \end{array} \right)$}.

Littlewood-Offord (1943) showed that $\rho(a):= O(n^{-1/2}\log{n})$.\\
Theorem (Erd\H{o}s,1945)

$\rho(a) \leq \frac{\left( \begin{array}{c} n \\ {\lfloor n/2 \rfloor} \end{array} \right)
    }{2^n} = O(n^{-1/2})$.
Note that the left inequality is tight by the example we discussed.\\
Proof:\\

Fix $x \in \mathbb{Z}$ W.l.o.g, we may assume that $a_i > 0$ for all $i \in [n].$ Let $\mathcal{A}$ denote the collection of all $A \subseteq [n]$ such that $\sum_{x \in A} a_i - \sum_{j \in A^c} a_j = x$. Since $a_i >0$ for all $i \in [n]$, $\mathcal{A}$ is an anti-chain. Hence, by Sperner's lemma, $|\mathcal{A}| \leq \left( \begin{array}{c} n \\ {\lfloor n/2 \rfloor} \end{array} \right).  \square$

For every $ \epsilon > 0$, there exists $C_\epsilon$ depending on $\epsilon$ such that for all sufficiently large n, and for all $t \geq 0$, $\mathbb{P}[S_n(Q_n)\leq t/\sqrt{n}] \leq C_\epsilon t +(1/2 +\epsilon)^n$
\subsection{Techniques}

We use the high-level strategy of dividing the unit sphere into 'structured' and 'unstructured' components, and estimating the contribution of each part separately.
However, this approach faces challenges, particularly in ensuring the invertibility of random matrices.
In the analysis of structured vectors, it's necessary to consider scenarios where rows or columns may be equal, whereas previous studies typically only addressed cases where a single row or column was zero.
Obtaining such estimates, even for simpler cases like the Boolean slice, is a complex task tackled in recent work by Litvak and Tikhomirov using the concept of 'UDLCD'.

\subsection{Structured vectors:}
\textbf{Overall Goal:} Analyze structured vectors' contribution to singularity probability in random matrices.\\
\textbf{Types of Structured Vectors:}

$\bullet${Elementary: Close to standard basis vectors (e.g., close to e1).}

$\bullet${Non-elementary: Not close to any single basis vector but still exhibit some structure.}\\
\textbf{Steps:}

$\bullet${\textbf{Develop novel anti-concentration estimates:} Propositions: Propositions 5.2[Fix a discrete distribution $\xi$ and $\delta \in (0, 1/2)$. There exists $\theta = \theta(\delta, \mathcal{E}) > 0$ such that for all $x \in \mathbb{S}^{n-1}\ Elem'(\delta) $, $\mathfrak{L}_\mathcal{E}(b_1x_1+\cdots+b_nx_n,0)\leq ||\overrightarrow{p}||_2^2-\theta]$ 
and 5.3 [the only difference between 5.2 and 5.3 is that in 5.3 they also add: 'Suppose that $\xi$ is not a translate of any origin-symmetric distribution.'] for non-elementary vectors.}

$\bullet${\textbf{New technique for analyzing elementary vectors:} Using the example of $e_1$, demonstrates that a small image leads to either a zero first column or a specific subset with low probability.}

$\bullet${\textbf{Expansion to general distributions:} Extend analysis to other elementary vectors (close to $(e_i \pm e_j)/\sqrt{2}$ or $e_i$) using rotations and preliminary estimates.}\\
Current Limitation: Missing a sharp analysis for unstructured vectors when the distribution is uniform.\\
Overall Significance: This research significantly advances the understanding of how structured vectors influence the invertibility of random matrices, paving the way for further analysis of unstructured cases.
\subsection{Unstructured vectors:}
Unlike "structured" vectors with similar components, "unstructured" vectors in random matrices have diverse values. The authors exploit this non-uniformity to analyze them. They introduce a novel "multi-slice" theorem to handle these vectors, overcoming challenges of dependence and non-integer values. This method offers a powerful tool for understanding how unstructured vectors influence the invertibility of random matrices, building upon previous work on simpler cases.
\section{Notation}
The notation defines key terms like unit vectors, Euclidean balls, norms, and specific sets used in the analysis.
Additionally, it clarifies the meaning of a discrete random variable and its corresponding random matrix, which are central concepts for the research presented.

\end{document}

