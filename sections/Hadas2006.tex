


~\cite{costello2005random}

The article "Random Symmetric Matrices Are Almost Surely Non-Singular" by K. Costello, T. Tao, and V. Vu, presents a significant result in the field of random matrices. The main finding of the paper is the proof that a symmetric random matrix \( Q_n \) with independent (identically distributed) Bernoulli random variables with upper diagonal entries is almost surely non-singular, with a probability of  \( 1-O(n^{-1/8+\delta}) \) for any \( \delta>0 \). This result extends previous results for random matrices to more general models of random matrices.

The article presents the history of the non-singularity problem in random matrices, namely whether it is true that a random matrix \( A_n \) with independent Bernoulli variables is almost surely non-singular. This question was positively answered by Komlós in 1967, and later he generalized the result to more general models of random matrices. In a recent paper, Tao and Vu found a different proof for random matrices that provides a precise estimate for the absolute value of the determinant of the matrix \( A_n \).

The authors build on these previous proofs and develop a quadratic version of Littlewood-Offord's proof results concerning the convergence of random variables to prove the non-singularity (invertibility) of \( Q_n \) - a symmetric random matrix with independent (identically distributed) Bernoulli random variables. This version helps researchers overcome the row and column transposition, which was a hurdle in previous proofs for random matrices due to the dependence between the row vectors of the matrix \( Q_n \).

In order to develop the quadratic version of Littlewood-Offord and prove it, the authors initially tried to invert the quadratic problem: \( Q=\sum_{1\leq i,j\leq n} c_{ij} z_i z_j \) for a linear problem: \( Q=\sum_{i=1}^n Q_i z_i \). The difficulty that arose is due to the fact that the coefficients \( Q_i \) depend on the variables \( z_i \). In order to overcome this problem, the authors used the "decoupling" technique.
\vspace{\baselineskip}


\textbf{Decoupling lemma:} Let \( X \) and \( Y \) be random variables and \( E=E(X,Y) \) be an event depending on \( X \) and \( Y \). Then:
\vspace{-0.6\baselineskip}
\[ P(E(X,Y)) \leq \left( P\left( E(X,Y) \land E(X',Y) \land E(X,Y') \land E(X',Y') \right) \right)^{1/4} \vspace{-0.5\baselineskip} \]

where \( X' \) and \( Y' \) are independent copies of \( X \) and \( Y \), respectively. This method is used in other fields, such as graph theory.
\vspace{\baselineskip}

The article raises open questions for future research in the field of random matrices:

\begin{itemize}
    \item \textbf{Determinant Estimation:} The article raises the question of estimating the determinant of random matrices. The estimation provided in the article is: \( |\det(Q_n)| = n^{1/2-o(1)} \).
    \item \textbf{Singularity Probability:} Another open question raised in the article relates to estimating the probability that a random matrix is singular. The authors estimate that the probability of \( Q_n \) being a singular matrix is \( (1/2+o(1))^n \).
\end{itemize}

\vspace{\baselineskip}
\textbf{The quadratic variant of the Littlewood-Offord:}
Let \( Q \) be a quadratic random variable defined as:
\vspace{-0.8\baselineskip}
\[ Q = \sum_{1 \leq i,j \leq n} c_{ij} z_i z_j \vspace{-0.5\baselineskip} \]
where \( z_i \) are random variables, \( \{1,\ldots,n\} = U_1 \cup U_2 \) is a non-trivial partition, and \( S \) is a non-empty subset of \( U_1 \). For each \( i \in S \), let \( d_i \) be the number of indices \( j \in U_2 \) such that \( |c_{ij}| \geq 1 \). If \( d_i \geq 1 \) for each \( i \in S \), and \( I \) is an interval of length 1, then:
\vspace{-0.9\baselineskip}
\[ P(Q \in I) = O\left( |S|^{-1/2} + |S|^{-1} \sum_{i \in S} d_i^{-1/2} \right)^{1/4} \]


