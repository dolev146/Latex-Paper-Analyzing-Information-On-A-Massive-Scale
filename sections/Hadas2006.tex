
\subsection{Random Symmetric Matrices Are Almost Surely Non-Singular}



"Random Symmetric Matrices Are Almost Surely Non-Singular" by Costello, Tao, and Vu (\cite{costello2005random}) proves that symmetric random matrices \(Q_n\) with i.i.d. Bernoulli entries above the diagonal are almost surely non-singular. The non-singularity holds with probability \(1-O(n^{-1/8+\delta})\) for any \(\delta>0\), broadening earlier findings to a wider class of random matrices.
The article delves into the history of non-singularity in random matrices, focusing on whether matrices like \(A_n\) with independent Bernoulli variables are almost surely non-singular. Komlós affirmed this in 1967 and later extended his findings. Tao and Vu, in a more recent work, introduced a novel proof offering a precise determinant estimate for \(A_n\), enriching the understanding of random matrix non-singularity.
Building upon prior work, the authors propose a quadratic Littlewood-Offord theory to ascertain \(Q_n\)'s non-singularity, a symmetric matrix with i.i.d. Bernoulli variables. This approach addresses challenges like row and column transposition dependency. Attempting to linearize the quadratic \( Q=\sum_{1\leq i,j\leq n} c_{ij} z_i z_j \) into \( Q=\sum_{i=1}^n Q_i z_i \) revealed dependencies among \( Q_i \) coefficients on \( z_i \). To navigate this, "decoupling" was employed, refining the theoretical framework for random matrices' invertibility analysis.
\textbf{Decoupling lemma:} Given random variables \(X\) and \(Y\), and an event \(E=E(X,Y)\) dependent on \(X\) and \(Y\), then:
\[
P(E(X,Y)) \leq (P(E(X,Y) \land E(X',Y) \land E(X,Y') \land E(X',Y')))^{1/4}
\]
where \(X'\) and \(Y'\) are independent copies of \(X\) and \(Y\), respectively. This technique finds application across various domains, including graph theory.
The article outlines open questions for future research in random matrices:
\begin{itemize}
    \item \textbf{Determinant Estimation:} It questions how to estimate the determinant of random matrices, providing an initial estimate: \( |\det(Q_n)| = n^{1/2-o(1)} \).
    \item \textbf{Singularity Probability:} It seeks to refine the estimation of the probability that a random matrix is singular, estimating \( Q_n \)'s singularity probability as \( (1/2+o(1))^n \).
\end{itemize}
\vspace{\baselineskip}
\textbf{The quadratic variant of the Littlewood-Offord:}
Let \( Q \) be a quadratic random variable defined as:
\vspace{-0.8\baselineskip}
\[ Q = \sum_{1 \leq i,j \leq n} c_{ij} z_i z_j \vspace{-0.5\baselineskip} \]
where \( z_i \) are random variables, \( \{1,\ldots,n\} = U_1 \cup U_2 \) is a non-trivial partition, and \( S \) is a non-empty subset of \( U_1 \). For each \( i \in S \), let \( d_i \) be the number of indices \( j \in U_2 \) such that \( |c_{ij}| \geq 1 \). If \( d_i \geq 1 \) for each \( i \in S \), and \( I \) is an interval of length 1, then:
\vspace{-0.9\baselineskip}
\[ P(Q \in I) = O\left( |S|^{-1/2} + |S|^{-1} \sum_{i \in S} d_i^{-1/2} \right)^{1/4} \]



