


~\cite{costello2005random} The article "Random Symmetric Matrices Are Almost Surely Non-Singular"
by K. Costello, T. Tao, and V. Vu presents a significant result in the
field of random matrices. The main finding of the article is the proof
that a random symmetric matrix $ Q_n $ with independent and identically distributed 
(with the same distribution) Bernoulli variables as its upper diagonal entries
is almost surely non-singular, with a probability of $ 1-O(n^{-1/8+\delta}) $  for any $ \delta > 0 $.
This result extends previous results for random matrices to more general models of random matrices.
The article presents the history of the non-singularity problem in random matrices,
namely whether it is true that a random matrix $ A_n $ with independent Bernoulli variables
is almost surely non-singular. This question was positively answered by Komlós in 1967,
and later he generalized the result to more general models of random matrices. In a recent paper,
Tao and Vu found a different proof for random matrices that 
provides a precise estimate for the absolute value of the determinant of the matrix $ A_n $.
Building upon these previous proofs, the authors develop a quadratic version of 
Littlewood-Offord type results concerning the concentration of random variables to prove 
the non-singularity of $ Q_n $- a  random symmetric matrix.
This method allows researchers to overcome the challenge of the row and column 
transpose, which was a hurdle in previous proofs for random matrices due to the 
dependence between the row vectors of the matrix $ Q_n $.
The article raises open questions for future research in the field of random matrices:

Determinant Estimation: The article raises the question of estimating the 
determinant of random matrices. The estimation provided in the article is: $ |det\ {Q_n|=}n^{\left(1/2-o\left(1\right)\right)n} $.

Singularity Probability: Another open question raised in the article relates to estimating the 
probability that a random matrix is singular. The authors estimate that
the probability of $ Q_n $ being a singular matrix is $ {(1/2+o\left(1\right))}^n $.

The quadratic variant of the Littlewood-Offord :
Let Q be a quadratic random variable defined as: $ Q=\sum_{1\le i,j\le n}{c_{ij}z_iz_j} $  where $ z_i $
are random variables, $ {1,\ldots,n}=\ U_1\cup U_2  $
is a non-trivial partition, and S is a non-empty subset of
$ U_1 $. For each $ i\in S $, let  $d_i$  be the number of indices  $j\in U_2$ 
such that $|c_{ij}|\geq1$. If 
$ d_i\geq1$ for each $i\in S$, and I is an interval of length 1, then:
$ P\left(Q\in I\right)=O{({|S|}^{-1/2}+{|S|}^{-1}\sum_{i\in S}{d_i}^{-1/2})}^{1/4} $
