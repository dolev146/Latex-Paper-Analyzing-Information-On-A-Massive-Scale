\newpage


\section*{ANTICONCENTRATION IN RAMSEY GRAPHS AND A PROOF OF THE
ERDŐS-MCKAY CONJECTURE}
The paper `ANTICONCENTRATION IN RAMSEY GRAPHS AND A PROOF OF THE
ERDŐS-MCKAY CONJECTURE`~\cite{kwan2022anticoncentration}

by Matthew Kwan, Ashwin Sah, Lisa Saurmann, and Methaab Sawhney
presents an examination of edge-statistics within C-Ramsey graphs,
contributing insights that advance our understanding toward
the resolution of the Littlewood-Offord problem. 
By analyzing the distribution of edges in random vertex subsets of these graphs,
we uncover new patterns that echo the fundamental tenets of the 
Littlewood-Offord theory, marking a notable progression.

Lets begin with a few basics. An induced subgraph of a graph is called homogeneous 
if it is a clique or independent set (i.e., all possible edges are present, or none are).
One of the most fundamental results in Ramsey theory,
proved in 1935 by Erdős and Szekeres~\cite{erdos1935combinatorial}  states that every 
n-vertex graph contains a homogeneous subgraph with
at least $ \frac{1}{2} \log_2{n} $ vertices.On the other hand, Erdős~\cite{erdos1947some} famously used
the probabilistic method to prove that, for all $n \geq 3$, there is an n-vertex graph
with no homogeneous subgraph on $2 \log_2{n}$ vertices. an upper bound and lower bound.

A C-Ramsey graph is an n-vertex graph without cliques or independent sets 
larger than \(C \log_2 n\), where \(C\) is a constant.


\subsection{\textbf{Theorem1}} Fix  $C > 0$  and $ \eta > 0$. Let  $G$  be a  C-Ramsey graph on  n  vertices,
where  n  is sufficiently large relative to  $C$  and  $\eta$. Then, for any integer  $x$  with
0 $\leq x \leq (1 - \eta)e(G)$,  there exists a subset  $U$ $\subseteq V(G)$  inducing exactly 
$x$  edges.
The theorem declares that for any large enough C-Ramsey
graph and integer within a specified range, there exists a vertex
subset inducing exactly that many edges.~\cite{alon2003induced}.


\subsection{\textbf{Edge statistics and low-degree polynomials}}
For a graph $G$ with vertex set $V(G) = \{1, \ldots, n\}$ and edge set $E(G)$,
subset of vertices $U \subseteq V(G)$ with a vector
$\mathbf{\xi} \in \{0,1\}^n$, where $\xi_i = 1$ if vertex $i$ is in $U$ and $\xi_i = 0$ otherwise.
The number of edges $e(G[U])$ in the induced subgraph $G[U]$ can then be represented
as the evaluation of a quadratic polynomial: $f(\mathbf{\xi}) = \sum_{\{i,j\} \in E(G)} \xi_i \xi_j$
where the sum runs over all edges in $E(G)$. 

the statement that G has an induced subgraph with exactly x edges is precisely equivalent to
the statement that there is a binary vector $\overline{\xi} \in \{ 0,1 \}^n $ with $f(\overline{\xi})=x $

Recent studies focus on random variables e(G[U]) for graphs G and random vertex sets U,
inspired by conjectures from Alon, Hefetz, Krivelevich, and Tyomkyn.~\cite{alon2020edge}.

\subsection{Theorem2}
\textit{Fix} $C, \lambda > 0$, \textit{let} $G$ \textit{be a} $C$-\textit{Ramsey graph on} $n$ \textit{vertices, and let} $\lambda \leq p \leq 1 - \lambda$. \textit{Then if} $U$ \textit{is a random subset of} $V(G)$ \textit{obtained by independently including each vertex with probability} $p$, \textit{we have}
$\sup_{x \in \mathbb{Z}} \Pr[e(G[U]) = x] \leq K_{C\lambda} n^{-3/2}$
\textit{for some} $K_{C\lambda} > 0$ \textit{depending only on} $C$ \textit{and} $\lambda$. \textit{Furthermore, for every fixed} $A > 0$, \textit{we have}

$
\inf_{x \in \mathbb{Z}, |x - p^2 e(G)| \leq An^{3/2}} \Pr[e(G[U]) = x] \geq \kappa_{C A \lambda} n^{-3/2}
$
\textit{for some} $\kappa_{C A \lambda} > 0$ \textit{depending only on} $C$, $A$, \textit{and} $\lambda$, \textit{if} $n$ \textit{is sufficiently large in terms of} $C$, $\lambda$, \textit{and} $A$.
In the contexts of this Theorem it's noted that
the distribution of the number of edges,
$e(G[U])$, in a subset $U$ of a graph $G$,
adheres to a central limit theorem [~\cite{berkowitz2018local},~\cite{berkowitz2016quantitative}, ~\cite{gilmer2016local},~\cite{gnedenko1948local} ], suggesting
$\Pr[e(G[U]) \leq x] = \Phi\left(\frac{x - \mu}{\sigma}\right) + o\left(\frac{1}{\sigma}\right)$,
with $\Phi$ representing the Gaussian CDF, and $\mu, \sigma$ the mean and standard deviation.
However, due to the degree sequence's additive structure,
a local central limit theorem, which would adjust to
$\Pr[e(G[U]) = x] = \frac{\Phi\left(\frac{x-\mu}{\sigma}\right)}{\sigma} +o\left(\frac{1}{\sigma}\right)$
using the Gaussian density function, does not generally hold.

\subsection{Small-ball probability for quadratic Gaussian chaos} 
The study of low-degree polynomials of independent random variables,
frequently referred to as chaoses,possesses noteworthy contributions within this domain.
for example in the paper of Kim-Vu polynomial concentration
~\cite{kim2000concentration}\dots. 
This area of research offers insights into the behavior and 
characteristics of such polynomials and described as the fundamental tools
in probabilistic combinatorics, high-dimensional statistics, the analysis
of boolean functions and mathematical modeling.
Much of this study has focused on low-degree polynomials of Gaussian random variables, which enjoy
certain symmetry properties that make them easier to study. While this direction may not seem obviously
relevant to the previous theorem,
part of the proof related to applying the celebrated 
Gaussian invariance principle of Mossel O'Donnell, and Oleszkiewicsz
~\cite{mossel2005noise}
to compare our random variables of interest with certain “Gaussian analogs”.
Hence, an essential phase in the demonstration of Theorem entails examining the small-ball
probability associated with quadratic polynomials of Gaussian random variables.
The Carbery-Wright theorem represents the foundational principle within this field of study
% ~\cite{carbery2001distributional}
Which says that for $ 0 < \epsilon < 1 $ and any real quadratic polynomial $f=f(Z_1,\dots,Z_n)$ 
of independent standard Gaussian random variable $Z_1,\dots,Z_n ~ \mathcal{N}(0,1)$ 
we have $\sup_{x \in \mathbb{R}} \Pr[|f - x| \leq \varepsilon] = O\left(\sqrt{\frac{\varepsilon}{\sigma(f)}}\right)$
or any quadratic polynomial of 
independent standard Gaussian variables and for any small epsilon,
the supremum of the probability that the polynomial's
deviation from any real value is within epsilon scales optimally
with epsilon over the standard deviation of the polynomial
Let $Z = (Z_1, \ldots, Z_n)$ be a vector of independent standard Gaussian random variables,
and consider a real quadratic polynomial
$f(Z) = Z^T F Z + f^T Z + f_0$, where $F$ is a nonzero symmetric matrix in
$\mathbb{R}^{n \times n}$, $f$ is a vector in $\mathbb{R}^n$, and $f_0$ is a real number.
Suppose that for some positive constant $\eta$, for any symmetric matrix $G$ of rank at most $2$,
the smallest eigenvalue of $F - G$ in the Frobenius norm is at least $\eta$. Then, for any $\epsilon > 0$ 
and any real number $x$, there exists a constant $C_{\eta}$, depending only on $\eta$, such that
$
\Pr\left( |f(Z) - x| < \epsilon \right) \leq \frac{C_{\eta} \epsilon}{\sigma(f(Z))},
$
where $\sigma(f(Z))$ denotes the standard deviation of $f(Z)$.

quadratic forms with robust rank 2, such as \( Z_1^2 - Z_2^2 \),
may not conform due to logarithmic scaling in their probability measures as \( \epsilon \)
approaches zero. 
Furthermore, the theorem is indicative of an inverse theorem,
stating that atypical small-ball behavior corresponds to the form's
closeness to a low-rank quadratic form, 
reflecting the insights from the Littlewood-Offord problem.
Kane's structure theorem~\cite{kane2017structure} further elaborates that quadratic polynomials
of Gaussian variables can be `decomposed' into parts with standard small-ball behavior,
facilitating a simpler analysis of their overall behavior.

Regarding Ramsey graphs adjacency matrices,
which are observed to have robustly high rank.
This property is related for applying the previous theorem,
in particular, a deeper examination into the rank
partitioning into submatrices (Lemma 10.1 in the paper).
The relationship between graph rank and the absence of large homogeneous sets suggests
intriguing implications for communication complexity,
specifically in the context of the log-rank conjecture.
Additionally, the research extends to demonstrate that binary matrices,
which approximate a low-rank real matrix,
closely mirror a low-rank binary matrix (Proposition 10.2 in the paper),
presenting potential for broader applications beyond Ramsey graph analysis.

Limitations in precisely determining edge counts \(e(G[U])\)
in Ramsey graphs via Fourier-analytic estimates,
introduces an averaged version of the \textbf{switching method.}
This method quantifies transitions between outcomes of events, focusing on small
perturbations within a random set \(U\). By examining moments of these transition
counts and applying the Cauchy-Schwarz inequality, the approach refines probability
estimates, potentially extending its application beyond Ramsey graphs.

