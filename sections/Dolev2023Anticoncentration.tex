
\subsection{Anticoncentration In Ramsey Graphs}\label{sec:anticoncentration}
Thea article written by Matthew Kwan, A.Sah, L.Saurmann, M.Sawhney 
\cite{kwan2022anticoncentration} investigates edge-statistics in C-Ramsey graphs,
in regards of the Littlewood-Offord problem.

Ramsey theory's cornerstone,
established by Erdős and Szekeres in 1935
\cite{erdos1935combinatorial},
asserts that every n-vertex graph contains a homogeneous subgraph of
at least $\frac{1}{2} \log_2{n}$ vertices, contrasting with Erdős' 1947 demonstration
\cite{erdos1947some} of an n-vertex graph lacking a homogeneous 
subgraph on $2 \log_2{n}$ vertices. A C-Ramsey graph is defined as an 
n-vertex graph without cliques or independent sets exceeding \(C \log_2 n\),
where \(C\) is a constant.

\subsubsection{Edge Statistics and Low-Degree Polynomials}
Consider a graph $G$ with vertices $V(G) = \{1, \ldots, n\}$ and edges $E(G)$. For a vertex subset $U \subseteq V(G)$, represented by $\mathbf{\xi} \in \{0,1\}^n$ where $\xi_i = 1$ indicates vertex $i \in U$, the edge count in $G[U]$ is given by a quadratic polynomial $f(\mathbf{\xi}) = \sum_{\{i,j\} \in E(G)} \xi_i \xi_j$. Thus, $G$ has an induced subgraph with $x$ edges if a binary vector $\overline{\xi}$ exists such that $f(\overline{\xi})=x$. This area, especially for random vertex subsets $U$, draws on conjectures by Alon et al.~\cite{alon2020edge}.


\subsubsection{Edge Distribution Tightness in Random Subsets of C-Ramsey Graphs}
Given a $C$-Ramsey graph $G$ on $n$ vertices and a probability $p$ in the range $\lambda \leq p \leq 1 - \lambda$, for a random vertex subset $U$ with inclusion probability $p$, it's found that $\sup_{x \in \mathbb{Z}} \Pr[e(G[U]) = x] \leq K_{C\lambda} n^{-3/2}$ for a constant $K_{C\lambda} > 0$, and $\inf_{x \in \mathbb{Z}, |x - p^2 e(G)| \leq An^{3/2}} \Pr[e(G[U]) = x] \geq \kappa_{C A \lambda} n^{-3/2}$ for $\kappa_{C A \lambda} > 0$, both dependent on $C$, $\lambda$, and additionally $A$ for the latter. This indicates the edge count $e(G[U])$ distribution approximates a Gaussian central limit with mean $\mu$ and standard deviation $\sigma$, albeit a precise local central limit theorem adjustment is not universally applicable due to the degree sequence's characteristics.


\subsubsection{Small-Ball Probability For Quadratic Gaussian Chaos}
This research area delves into low-degree polynomials of independent random variables,
notably Gaussian chaoses,
showing insights regarding the low-degree polynomials.
Central to this study is the Gaussian invariance principle,
A key aspect involves examining the small-ball probability for
quadratic Gaussian variables, highlighted by the Carbery-Wright theorem.
It establishes that for any real quadratic polynomial $f$ of
independent standard Gaussian variables and a small
$\epsilon$, 
the probability that $f$'s deviation from any value is within
$\epsilon$ scales with $\epsilon$ over $f$'s standard deviation.

Further, specific quadratic forms exhibit distinct
probability behaviors, shedding light on the 
Littlewood-Offord problem and suggesting a structure theorem
for simplifying analysis of Gaussian variable polynomials.
This theorem's implications extend to the study of Ramsey graphs,
especially in understanding the adjacency matrices' rank and its
implications for communication complexity and the 
log-rank conjecture.
The research also introduces an averaged version of the 
\textbf{switching method} to refine probability estimates
in Ramsey graph analysis,
potentialy applicability in broader contexts.


in the next article we dive into more implications of Ramsey graphs in the 
Algebraic inverse theorem and the relation with the quadratic Littlewood theorem.


