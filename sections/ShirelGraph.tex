
\subsection{Algebraic Inverse Theorem For The Quadratic Littlewood-Offord Problem}
% \subsubsection{Quadratic Approximation Bound}
in the article Algebraic Inverse Theorem For The Quadratic Littlewood-
Offord Problem \cite{kwan2019algebraic} bellow theorems are mentioned regarding low-degree polynomials.
% \textbf{Theorem 1.1} establishes that given $r \geq 3$, $0 < \epsilon \leq 1$, and a constant $C = C(r, \epsilon)$, for any quadratic polynomial $f \in F[x_1, \dots, x_n]$ (with $F$ being $\mathbb{C}$, $\mathbb{R}$, or $\mathbb{Q}$) having coefficients of absolute value at most 1, and $\xi = (\xi_1, \dots, \xi_n) \in \text{Rad}^n$, if ${\sup_{x \in F} \Pr(f(\xi) = x) \geq C \cdot \frac{(\log n)^{r/2}}{n^{1-2/(r+2)}}}$, then a quadratic form $h \in F[x_1, \dots, x_n]$ exists with rank < $r$ and the coefficient sum of $f - h$ within $\epsilon n^2$.
% =======
% The article introduces inverse theorems of a similar flavour to Costello's
% conjecture, showing that if a quadratic polynomial exhibits point
% probabilities significantly larger than ${1/n}$,
% it must be close to a low-rank quadratic form.
% This is achieved through a detailed analysis
% involving the arrangement of coefficients
% and the algebraic structure of the polynomial.

% The authors in the given paper focused on the quadratic
% Littlewood-Offord problem by examining the concentration
% of quadratic polynomials in independent Bernoulli random
% variables. They extended classical questions about linear
% polynomials to the quadratic case, exploring the
% conditions under which a quadratic polynomial can have
% significant point probabilities. Their main results,
% as summarized in Theorems 1.1 and 1.2, establish a connection
% between the anti-concentration of a quadratic polynomial 
% $f(\xi)$ and the algebraic structure of $f$.
% In particular,
% they demonstrated that if a quadratic polynomial has a 
% concentration probability significantly
% larger than $\frac{1}{n}$, it must be close to
% a quadratic form with low rank.

% \subsubsection{Quadratic Approximation Bound}
% In \textbf{Theorem 1.1}, for any given $r \geq 3$ and
% $0 < \epsilon \leq 1$, they identified a constant
% $C = C(r, \epsilon)$ such that for a quadratic polynomial
% $f \in F[x_1, \dots, x_n]$ (where $F$ is either 
% $\mathbb{C}$, $\mathbb{R}$, or $\mathbb{Q}$) with all coefficients
% of absolute value at most 1, and
% $\xi = (\xi_1, \dots, \xi_n) \in \text{Rad}^n$, if
% ${\sup_{x \in F} \Pr(f(\xi) = x) \geq C \cdot \frac{(\log n)^{r/2}}{n^{1-2/(r+2)}}}$,
% then there exists a quadratic form
% $h \in F[x_1, \dots, x_n]$ of rank strictly less than $r$
% such that the sum of the absolute values of the coefficients
% of $f - h$ is at most $\epsilon n^2$.
 
% Theorem 1.1 provides a significant breakthrough in this 
% area by demonstrating that if a quadratic polynomial shows 
% point probabilities significantly larger than $1/n$, then
% it closely resembles a quadratic form of low rank.
% This finding is profound because it not only generalizes
% the Littlewood-Offord problem to quadratic polynomials
% but also introduces an "inverse" aspect to the problem

\subsubsection{Constrained Coefficient Approximation}
\textbf{Theorem 1.2} explores quadratic polynomials ${f}$ with degree-2 coefficients 
from set $S$, showing that under anti-concentration conditions, a quadratic form $h$ 
nearly identical to ${f}$ (differing by at most $\epsilon n^2$ coefficients) can be 
found with rank less than $r$. This extends the Littlewood–Offord problem's scope to quadratic polynomials by analyzing their concentration probabilities and applying a structural constraint on the coefficients. The theorem enhances anti-concentration estimates for quadratic polynomials, particularly those resistant to factorization over $\mathbb{C}$, aligning with Costello's conjecture. It indicates that for polynomials $f(\xi)$ with point probabilities above $n^{-3/5}$, a closely related quadratic form $h$ exists that factorizes into linear factors in $\mathbb{C}$. Additionally, the paper applies these insights to Ramsey graphs, offering asymptotic solutions to previously posed questions by Kwan, Sudakov, and Tran.


\subsubsection{Edge Anti-Concentration in C-Ramsey Graphs}
In C-Ramsey graphs, as discussed at~\ref{sec:anticoncentration}, 
there is an expectation for these graphs to exhibit characteristics akin to random graphs,
including the pioneering work by Erdos and Szemerédi~\cite{erdHos1972ramsey}.
This line of inquiry extends into examining the edge distribution's anti-concentration
within Ramsey graphs, as questioned by Kwan, Sudakov, and Tran~\cite{kwan2019anticoncentration}.
They inquire if for an n-vertex C-Ramsey graph and a uniformly random subset of
$n/2$ vertices, the probability $Pr(X=x)=O(1/n)$ holds for all $x \in \mathbb{N}$,
challenging previous notions of edge distribution concentration.

\textbf{Theorem 1.3} asserts for any $n$-vertex $C$-Ramsey graph and a random subset of vertices, the probability $\Pr(X = x) \leq n^{o(1)-1}$ for any $x \in \mathbb{Z}$. This demonstrates a significant anti-concentration in edge counts, indicating a distribution akin to that observed in random graphs, and applies insights from quadratic polynomial behavior to Ramsey graphs.
The theorem highlights the importance of algebraic methods in graph theory,
especially for analyzing Ramsey graphs' randomness and structure with algebraic probability.

