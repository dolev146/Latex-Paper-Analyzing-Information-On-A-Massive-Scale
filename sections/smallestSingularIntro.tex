In the study of random matrix theory within linear algebra, the concept of singular values, particularly the smallest singular value denoted as $s_n(A)$, is fundamental for understanding matrix behavior under perturbations. Singular values of a square $n \times n$ matrix $A$ are defined as the square roots of the eigenvalues of both $A^TA$ and $AA^T$, where $A^T$ is the transpose of $A$. These values are inherently non-negative and are traditionally arranged in non-increasing order ($s_1(A) \geq s_2(A) \geq \cdots \geq s_n(A)$), with $s_n(A)$ being the smallest. The focus on $s_n(A+M)$, where $M$ represents a matrix of random noise added to a deterministic matrix $A$, is crucial. It allows for the assessment of the impact of noise on the matrix's singular value spectrum, especially the smallest singular value. This smallest singular value, $s_n$, serves as an indicator of a matrix's proximity to singularity or non-invertibility. A lower $s_n$ signifies that the matrix is nearing a determinant of zero, hence closer to singularity, while a higher $s_n$ indicates a well-conditioned matrix, distant from being singular. This understanding is vital for evaluating the stability and invertibility of matrices in the presence of randomness or noise.
