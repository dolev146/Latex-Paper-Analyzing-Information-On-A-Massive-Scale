In the context of the singularity of random matrices, the smallest singular value, denoted as $s_n(A)$ for a matrix $A$, plays a crucial role. Singular values of a square $n \times n$ matrix $A$ are the square roots of the eigenvalues of $A^TA$, where $A^T$ is the transpose of $A$. These singular values are always non-negative and are often arranged in non-increasing order, so $s_1(A) \geq s_2(A) \geq \cdots \geq s_n(A)$. Here, $s_n(A)$ represents the smallest singular value.\\\\
The significance of the smallest singular value lies in its ability to provide insight into the matrix's stability and sensitivity to perturbations. Specifically, a matrix with a very small singular value is close to being singular (non-invertible), indicating that small changes in the matrix or in a linear system involving the matrix can lead to large changes in solutions or outputs. This concept is intimately related to the condition number of the matrix, defined as $\kappa(A) = \frac{s_1(A)}{s_n(A)}$, which measures how much the output value of a function can change for a small change in the input argument. A high condition number signifies potential loss of precision and numerical instability in calculations involving the matrix.\\\\
In the study of random matrices, the distribution and behavior of the smallest singular value among various classes of random matrices are of significant interest. This analysis aids in understanding the invertibility of these matrices and their robustness to perturbations, impacting numerical methods, signal processing, and data science, among other fields. Establishing bounds on the probability that the smallest singular value of a random matrix falls below a certain threshold is particularly relevant. Such bounds offer insights into the likelihood of a random matrix being nearly singular or well-conditioned, thereby affecting the reliability of computations performed with these matrices.