\subsection{Bilinear and Quadratic variants on the Littlewood-Offord problem}

Costello's paper~\cite{costello2009bilinear} presents extensions of the Littlewood-Offord problem and related results to polynomials of higher degree, in particular bilinear and quadratic forms.

They presented the history regarding the problem in its linear form specifically noting Halász's extension, which bounds the probability that the sum of vectors in \(R^d\), each multiplied by independent complex-valued random variables, equals a specific value, under the condition that no proper subspace contains too many of these vectors.

They highlighted the work of Kahn, Komlos, and Szemeredi, which utilized the concept to demonstrate that the singularity probability of a matrix is exponentially small relative to its size. Furthermore, Tao and Vu, along with Rudelson and Vershynin, confirmed this observation by showing that if the sum assumes a single value with a probability of at least \(n^{-c}\) for some fixed \(c\), then the coefficients must originate from a short generalized arithmetic progression.
\newline
\vspace{-0.6\baselineskip}

Their first main result shows is that every bilinear form with sufficiently large concentration probability is in some sense close to this degenerate example.\\
\vspace{-0.6\baselineskip}

The theorem establishes that for bilinear forms \(x^T A y\) with high concentration probabilities, where \(x\) and \(y\) are random vectors with entries chosen from \(\{1, -1\}\), the coefficient matrix \(A\) must contain a large rank-one submatrix if each row has a sufficient number of nonzero entries. This condition holds true even when the function \(f(y)\), determining the concentration level, is constant or when \(y\)'s entries have probabilities altered to include zeros.\\
\vspace{-0.6\baselineskip}

The next theorem addresses the concentration of quadratic forms \(x^T A x\), where \(x\) is a vector of random variables and \(A\) is a symmetric matrix with a significant number of nonzero entries in each row. It establishes that the probability of the quadratic form equaling a linear form \(L(x)\) plus a constant \(c\) is bound by a function of \(r\), the minimum number of nonzero entries per row, suggesting a dispersion characteristic that becomes more pronounced as \(r\) increases. This theorem applies even when the linear form \(L(x)\) is zero, illustrating the broad applicability of the result to understanding the dispersion of quadratic forms involving random variables.\\
\vspace{-0.6\baselineskip}

Costello's extensions provide as the basis to further expansion of the understanding of quadratic variants of the Littlewood-Offord problem.