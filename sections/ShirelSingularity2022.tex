

\begin{flushleft}
~\cite{campos2020singularity}
The authors improve upon previous results regarding the singularity probability of $M_n$, an $n \times n$ symmetric matrix with entries uniformly chosen from $\{\pm1\}$. They show that the probability $M_n$ is singular is at most $\exp(-c(n \log n)^{1/2})$. This result is an advancement over the previously best-known bound of $\exp(-cn^{1/2})$ Campos, Mattos, Morris and Morrison.\\\newline
The article discusses the singularity of random matrices, specifically, it addresses the singularity probability of matrices with entries from $\{-1, 1\}$, highlighting a conjecture that the probability a random $n \times n$ matrix $A_n$ is singular equals $(1 + o(1))n2^{-n+1}$, based on the likelihood of two rows or columns being identical up to sign.\\\newline
While this conjecture has seen progress over the years, with bounds on the singularity probability being improved through various mathematical advances, the paper's focus shifts to symmetric matrices. For symmetric matrices $M_n$, it's believed the singularity probability behaves similarly to the non-symmetric case, but less progress has been made. Earlier results have only shown that this probability tends to zero, without providing tight bounds.\\\newline
This paper not only offers tighter bounds on the singularity probability of $M_n$ but also argues that their method simplifies the approach taken by previous research. Specifically, they highlight an improved and simpler version of the "rough" inverse Littlewood-Offord theorem, which is crucial to their analysis. The introduction then elaborates on the techniques such as the inverse Littlewood-Offord Theorems, Fourier Analysis, Cauchy-Davenport Inequality, etc. involved in proving their main result, illustrating the complexity of dealing with "structured" versus "unstructured" vectors in the analysis of $M_n$'s singularity.\\\newline
\end{flushleft} 
