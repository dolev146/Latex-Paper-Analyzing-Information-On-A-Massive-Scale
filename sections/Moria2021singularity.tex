~\cite{jain2021singularity}
The article investigate the singularity probabilities of n $\times$ n discrete random matrices ${M_n}(\xi)$, with entries are non-constant real-valued random variables $\xi$. 
It explores the probability of singularity $ \mathbb{P}[M_n$ is singular$]=\mathbb{P}[$zero row or column$]+ (1+o_n(1))\mathbb{P}[$two equal (up to sign) rows or columns$]$ in these random matrices and confirm a conjecture related to this topic. 
Special attention is given to structured and unstructured vectors in relation to matrix invertibility of these matrices, providing precise singularity probabilities for Bernoulli distributions across different ranges of $\rho$:\\
For Bernoulli distributions with $\rho$ in the ranges $(0,1/2)$ and (1/2,1), the singularity probabilities are calculated as follows:
\begin{itemize}
    \item \textbf{Example for Bernoulli Distribution with $\rho \in (0, 1/2)$}: \\
    $\mathbb{P}[M_n \text{ is singular}] = 2n(1-\rho)^n + (1+o_n(1))n(n-1)(\rho^2+(1-\rho)^2)^n$.
    \item \textbf{For Bernoulli Distribution with $\rho \in (1/2, 1)$}: \\
    $\mathbb{P}[M_n \text{ is singular}] = (1+o_n(1))n(n-1)(\rho^2+(1-\rho)^2)^n$.
\end{itemize}
Building on the works of Litvak and Tikhomirov ~\cite{litvak2022singularity} and Tikhomirov ~\cite{tikhomirov2020singularity}, novel techniques for handling structured vectors are developed, distinguish between elementary and non-elementary structured vectors, to advance our understanding of their impact matrix invertibility.
One significant aspect of the study involves analyzing the contribution of the 'compressible' part of the unit sphere into 'structured' and 'unstructured' components to the lower tail of the smallest singular value of the matrices. By examining this contribution, the researchers aim to deepen the understanding of processes occurring in these random matrices and identify the impact of specific subsets on the singular behavior of the matrices.
The study innovates with a "multi-slice" theorem to analyze "unstructured" vectors, enhancing the analytical framework for dissecting the singular behavior of random matrices.
