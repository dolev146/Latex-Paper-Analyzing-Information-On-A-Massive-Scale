\textbf{The authors} of the article On the smallest singular value of symmetric random matrices ~\cite{jain2020smallest} extended the work on the smoothed analysis of the smallest singular value ~\cite{jain2020smoothed}, particularly in contexts involving discrete noise. 
Their research builds upon and broadens previous findings by offering new insights into how the smallest singular value of a matrix behaves when perturbed by noise, including discrete types. 
Their contributions include relaxing the conditions required for analyzing the stability of matrices under perturbations and introducing novel methods to study the effects of noise, thereby enhancing the theoretical framework that underpins the smoothed analysis in numerical linear algebra.\\
Their work notably advances the understanding of matrix perturbations beyond the contributions of earlier researchers such as Rudelson and Vershynin, who had established foundational results in the area of random matrices and their singular values. By focusing on less restrictive conditions and exploring the role of discrete noise, Jain, Sah, and Sawhney not only refine the bounds on the smallest singular value but also elucidate the practical implications of their findings for the performance and robustness of numerical algorithms. This relationship between their work and that of their predecessors underscores a significant progression in the field, driving forward the applicability of smoothed analysis to a wider range of problems and algorithms.\\\\
