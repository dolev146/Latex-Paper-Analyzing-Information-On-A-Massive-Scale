~\cite{jain2020smallest} 
This study examines the smallest singular value of n $\times$ n random symmetric matrices $A_n(ij) = A_n(ji)$, with entries as independent copies of a sub-Gaussian random variable $\xi$ (mean 0, variance 1).
The probability that $S_n(A_n) \leq \epsilon\sqrt{n}$ is bounded by:
\begin{equation*}
    \mathbb{P}[S_n(A_n) \leq \epsilon \sqrt{n}] \leq O_\xi(\epsilon^{1/8} + \exp(-\Omega_\xi(n^{1/2}))) \text{ for all } \epsilon \geq 0.
\end{equation*}
This improves a Vershynin result, extending it to a broader class of random variables.
The smallest singular value $s_n(M_n)$ of $M_n$ is defined as: $s_n(M_n) = inf_{v \in \mathbb{S}^{n-1}} \|Mv\|_2$
where $\mathbb{S}^{n-1}$ represents the unit sphere in $\mathbb{R}^n$ ~\cite{rudelson2008littlewood}.
For a random symmetric matrix $A_n$ with sub-Gaussian entries, the probability that $s_n(A_n) \leq \epsilon/\sqrt{n}$ is bounded by:
\begin{equation*}
    \mathbb{P}[s_n(A_n)\leq \epsilon/\sqrt{n}]\leq C \epsilon^{1/8}+2e^{-c n^{1/2}} \text{ for all } \epsilon \geq 0
\end{equation*}
where $C$ and $c$ are constants that depend on the sub-Gaussian norm of $\xi$.
When $\xi$ is a Rademacher random variable, the probability bound is:
\begin{equation*}
    P[s_n(A_n) \leq \epsilon / \sqrt{n}] \leq O(\epsilon^{1/8}+ \exp(-\Omega((\log{n})^{1/4}n^{1/2})))
\end{equation*}
This article introduces the Median Regularized Least Common Denominator (MRLCD) and the Median Threshold, enhancing the Regularized Least Common Denominator (RLCD) by leveraging vector arithmetic structure.
These concepts offer improved quantitative estimates and can replace RLCD in various applications.\\
The methods include probabilistic techniques, concentration inequalities, and analysis of random matrix theory, providing systematic analysis of singularity properties and tight bounds on singular values.
Challenges include optimizing probability bounds, extending results to different random variables, and refining the analysis for random symmetric matrices.\\
The article mentions two prominent examples: Ginibre Ensemble, where entries are independent Gaussian random variable, and i.i.d. Rademacher Matrices, with entries being Rademacher random variable ~\cite{bourgain2010singularity}.\\
These examples serve as fundamental models for understanding random matrix behavior under different distributions, facilitating probabilistic bound derivations.