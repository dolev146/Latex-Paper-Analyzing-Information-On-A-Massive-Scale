~\cite{jain2020smallest}
In this article the authors explores the behavior of the smallest singular value in n $\times$ n random symmetric matrices  $A_n(ij) = A_n(ji)$. 
They investigate $M_n$ random matrix $n\times n$ each of the entries is an independent copy of a sub-Gaussian random variables $\xi$ with mean $0$ and variance $1$.
The smallest singular value of $M_n$ is denoted as $s_n(M_n)$, defined as: $s_n(M_n) = inf_{v \in \mathbb{S}^{n-1}} \|Mv\|_2$ where $\mathbb{S}^{n-1}$ is the unit sphere in $\mathbb{R}^n$ \textbf{(from The Littlewood-Offord problem and invertibility of random matrices)}.
The main result shows that for a random symmetric matrix $A_n$ with sub-Gaussian entries, the probability of $s_n(A_n)$ being less than $\epsilon/\sqrt{n}$ is bounded by: $P[s_n(A_n)\leq \epsilon/\sqrt{n}]\leq C \epsilon^{1/8}+2e^{-c n^{1/2}}$ for all $\epsilon \geq 0$, where $C$ and $c$ are constants depending on the sub-Gaussian norm of $\xi$.
When $\xi$ is a Rademacher random variable, the probability bound becomes: $P[s_n(A_n) \leq \epsilon / \sqrt{n}] \leq O(\epsilon^{1/8}+ \exp(-\Omega((\log{n})^{1/4}n^{1/2})))$
It also mentions the Median Regularized Least Common Denominator (MRLCD) and the Median Threshold.
These notions improve upon the Regularized Least Common Denominator (RLCD) by efficiently utilizing the arithmetic structure of vectors, particularly when many projections of a vector are arithmetically unstructured.
They demonstrate that the MRLCD and median threshold have level sets that can be covered by sufficiently small nets at the appropriate scale, which is crucial for their applications. 
These new concepts can replace RLCD in various applications and are expected to provide better quantitative estimates. 
 