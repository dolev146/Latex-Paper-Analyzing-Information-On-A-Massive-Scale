\subsection{Resolution of the quadratic Littlewood--Offord problem}


In the article "Resolution of the quadratic Littlewood--Offord problem"~\cite{kwan2023resolution} by M. Kwan and K. Sauremann, the authors claim to have succeeded in finding a unique value for the quadratic polynomial $Q(\xi_1,\ldots,\xi_n)$, where $\xi_1,\ldots,\xi_n$ are independent Rademacher random variables.
 
Specifically, if $Q(\xi_1,\ldots,\xi_n)$ strongly depends on at least $m$ of the variables $\xi_i$, such that there is no way to determine the value of $Q(\xi_1,\ldots,\xi_n)$ by fixing values for fewer than $m$ of the variables $\xi_i$, then:
$\Pr[Q(\xi_1,\ldots,\xi_n) = 0] \leq O(1/\sqrt{m})$. A similar result holds when $\xi_1,\ldots,\xi_n$ have arbitrary distributions.
 
The proof combines several independent ideas, including an inductive decoupling technique that reduces the quadratic anticoncentration problem to a multi-dimensional linear anticoncentration problem.
\newline
The main result contributes to results in other areas (such as graph inducibility).
 
Furthermore, in this article, the authors achieve an essentially optimal bound for the quadratic Littlewood–Offord problem, as conjectured by Nguyen and Vu. The quadratic version of the Littlewood–Offord problem gained popularity after the work of Costello, Tao, and Vu on symmetric random matrices.
 
Several fundamental theories in probability theory focus on concentration of inequalities, demonstrating that certain types of random variables tend to concentrate around their mean (in a small interval). Conversely, the anti-concentration of inequalities sets the upper limit on the probability that random variables fall within a small range or even become identical to a certain value. In the realm of anti-concentration, one central problem is the Littlewood–Offord polynomial problem.
 
Roughly defined, the problem is as follows: Let $P$ be a polynomial with $n$ variables such that $P \in \mathbb{R}[x_1,\ldots,x_n]$, and consider them to be independent and random according to Rademacher, meaning the variables $\xi_1,\ldots,\xi_n \in \{1,-1\}$ with a distribution of $1/2$. The authors investigate what is the maximum $z$ given $\Pr[P(\xi_1,\ldots,\xi_n) = z]$, without assuming strong hypotheses about the polynomial $P$.
 
The authors present the historical development of the problem:
\newline
In 1943, Littlewood and Offord proved that for the linear case, i.e., when the polynomial $P$ is of degree 1:
\vspace{-0.4\baselineskip}
\[ \sup_{z\in\mathbb{R}} \Pr[X=z] \leq O(\log n/\sqrt{n}) \vspace{-0.8\baselineskip} \]
\newline
Where $X$ is a random variable, defined as follows: $X = a_i \xi_i + \ldots + a_n \xi_n$.
\newline
Their result received significant attention and was sharpened by Erdős in 1945, who found a purely combinatorial proof, which is known as the theory of Erdős Littlewood–Offord. Under the same assumptions:
\vspace{-0.2\baselineskip}
\[ \sup_{z\in\mathbb{R}} \Pr[X=z] \leq \left( n \choose \lfloor n/2 \rfloor \right) \cdot 2^{-n} = O(1/\sqrt{n}) \vspace{-0.5\baselineskip} \]
\newline
This result, obtained in the case where all coefficients $a_i$ are equal, is the best possible.
 
The linear Littlewood–Offord problem has been extensively studied, leading to investigations into the quadratic Littlewood–Offord problem. A question arises: what are the limits, we can find for the concentration point of the quadratic polynomial $Q(\xi_1,\ldots,\xi_n)$ when $\xi_1,\ldots,\xi_n$ are independent and random Rademacher variables? This question came to the forefront of research in 2005 in a paper by K. P. Costello, T. Tao, and V. Vu ("Random Symmetric Matrices Are Almost Surely Non-Singular"), where they used such a limit in their proof of Weiss's conjecture on random singular matrices, specifically proving that if the quadratic polynomial $Q \in \mathbb{R}[x_i \ldots x_n]$ has at least $cn^2$ coefficients that are non-zero for a constant $c > 0$, then for $X = Q(\xi_1,\ldots,\xi_n)$, where the $\xi_i$ are independent and random Rademacher variables:
\vspace{-0.24\baselineskip}
\[ \sup_{z\in\mathbb{R}} \Pr[X=z] \leq O(1/n^{1/8}) \vspace{-0.27\baselineskip} \]
Previously, K. P. Costello, T. Tao, and V. Vu, as well as A. Ferber, V. Jain, and Y. Zhao, have recognized that this result was not optimal and one can expect to achieve a similar limit to that of the linear Littlewood–Offord problem.
 
The first improvement was made by Costello and Vu ("The rank of random graphs"), showing how to prove that the limit is of the form $O(n^{-1/4})$. Further improvement, almost optimal, was achieved by Costello in the paper "Bilinear and quadratic variants on the Littlewood-Offord problem" to $O(n^{-1/2+\varepsilon})$ for any constant $\varepsilon > 0$. Through a different method, Meka, Nguyen, and Vu further extended the bound to $\exp(O(\log\log n)^2))/\sqrt{n}$, before it was observed that the bound $(\log n)^{O(1)}/\sqrt{n}$ follows from a strong general result by Kane.
 
In this article, we finally solve the quadratic problem of the Littlewood–Offord problem up to constant factors, reaching the optimal limit of $O(1/\sqrt{n})$. We also utilize weaker assumptions about $Q$.
 
The article focuses on proving the theory presented above. Additionally, it presents a generalization of this theory and explains how to derive this specific case of Rademacher variables, from the general case, where $\xi_1,\ldots,\xi_n$ are distributed arbitrarily so that any discrete random variable can be realized as a Rademacher random variable.
 
The main theory: We define $Q(\xi_1,\ldots,\xi_n)$ to be a polynomial of maximum degree 2, and let $\xi_1,\ldots,\xi_n$ be independent Rademacher random variables. If the polynomial $Q$ strongly depends on at least $m$ non-zero variables $\xi_i$, then:
\vspace{-0.36\baselineskip}
\[ \sup_{z\in\mathbb{R}} \Pr[Q(\xi_1,\ldots,\xi_n) = z] \leq \frac{C}{\sqrt{m}} \vspace{-0.5\baselineskip} \] for some constant $C$.
 
Until now, there have been several different proofs of the bound $O(1/\sqrt{n})$ in the linear Littlewood–Erdős theory. As far as is known, all these proofs exploit at least one of two special properties of random variables of the form $X = a_i \xi_i + \ldots + a_n \xi_n$. First, the original proof by Erdős ("On a lemma of Littlewood and Offord") uses monotonicity of $X$, and secondly, it leverages the fact that $X$ is a sum of independent Rademacher variables (in a simple distribution) in such a way that the Fourier transformation behaves very well.
 
Unfortunately, in the quadratic problem (where $X$ is the quadratic polynomial $Q$), both of these properties fail dramatically. There are two general approaches that have been successful so far: Gaussian approximation and the decoupling technique.
 
In this article, we take a different perspective on the decoupling technique. Instead of directly using decoupling to reduce the quadratic problem to a linear one, we reinterpret decoupling as a tool to create a problem with both linear and quadratic components. Inductively, we aim to "reduce the proportion of our problem that is quadratic" gradually to a linear form. To explain this concept, the authors rely on a geometric perspective.
 
*****The proof, as outlined, relies on bounds on probabilities that random vectors lie in certain affine-linear subspaces. More specifically, for a suitably nondegenerate affine-linear subspace $W \subseteq \mathbb{R}^n$ of codimension $k$, and a uniformly random vector $\vec{\xi} \in \{1,-1\}^n$, we need a probability bound of the form $\Pr[\vec{\xi} \in W] \leq O(n^{-k/2})$. Intuitively, this is because $\vec{\xi}$ needs to simultaneously satisfy $k$ different linear equations, each of which is satisfied with probability roughly $n^{-1/2}$. More formally, such a bound follows from a high-dimensional version of the Erdős–Littlewood–Offord theorem.
 
The first such high-dimensional version was due to Halász, and several extensions and variants of Halasz’ inequality have since been proved.
 
Of course, whenever one wants to apply any Halász-type theorem, one needs a “robust rank” condition to hold. So, in order to execute the strategy described, at each step of the decoupling scheme we need a “robust rank inheritance” lemma, proving that a robust rank condition is likely to hold for the next step, given that it holds for the current step. The key ingredient for our robust rank inheritance lemma is a new high-dimensional anticoncentration inequality for the probability that a random vector falls in a small ball in the Hamming norm. We believe this inequality (and the techniques in its proof) to be of independent interest.
 
The authors remark that there is a way to sidestep the robust rank inheritance issue in the special case where $Q$ has “bounded rank”, meaning that the quadratic part of $Q$ can be written as $\vec{x}^T A \vec{x}$ for some symmetric matrix $A$ of rank $O(1)$. Indeed, in this case we can reduce our entire problem to a certain bounded-dimensional geometric anticoncentration problem.******
 
\vspace{\baselineskip}
The article identifies directions for future research, such as:
 
\begin{enumerate}
	\item \textbf{Spherical Concentration}: The theory presented focuses on point concentration. It would be interesting to explore whether instead of point concentration, we can extend the theory to spherical concentration. That is, under what assumptions for $Q$, can we prove:
	\vspace{-0.3\baselineskip}
	\[ \sup_{z\in\mathbb{R}} \Pr[|Q(\xi_1,\ldots,\xi_n) - z| \leq 1] \leq O(1/\sqrt{n})? \vspace{-0.4\baselineskip} \]
	
	\item \textbf{Generalization beyond Degree 2}: It would be intriguing to generalize the theory beyond degree 2, i.e., to focus on polynomials of higher degrees and examine their properties in the context of point concentration.
	
	\item \textbf{Geometric Theory Generalization}: It would be desirable to generalize our geometric theory to geometric objects other than the square object.
	
	\item \textbf{Utilizing techniques for advancing weak points in Gotsman–Linial Estimation}: It would be interesting to investigate if techniques used in the paper can help us advance weak points in Gotsman–Linial Estimation.

	\item \textbf{Inverse Theory for the Square Case}: For the two problems of Littlewood–Offord, the linear and the quadratic, one cannot expect a stronger limit than $O(1/\sqrt{n})$. However, it would be natural to investigate certain assumptions, could indeed yield stronger bounds.
	
	Despite the significant success in the linear case, which led to the development of the inverse theory by Tao Vu in the article "Inverse Littlewood–Offord theorem," the square case is less known. It would be interesting to explore whether it is possible to develop such inverse theory also for the square case and what its optimality would be.
\end{enumerate}
 
These are interesting directions that could enhance and broaden the field, leading to new developments in future research.
